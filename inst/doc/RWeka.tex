\documentclass[fleqn]{article}
\usepackage[round,longnamesfirst]{natbib}
\usepackage{hyperref}
\usepackage{a4wide}

\newcommand{\pkg}[1]{{\normalfont\fontseries{b}\selectfont #1}}
\newcommand{\file}[1]{`{\textsf{#1}}'}

%% \VignetteIndexEntry{RWeka Odds and Ends}
\title{RWeka Odds and Ends}
\author{Kurt Hornik}

\usepackage{Sweave}
\begin{document}


\maketitle

\pkg{RWeka} is an R interface to Weka \citep{RWeka:Witten+Frank:2005}, a
collection of machine learning algorithms for data mining tasks written
in Java, containing tools for data pre-processing, classification,
regression, clustering, association rules, and visualization.  Building
on the low-level R/Java interface functionality of package \pkg{rJava}
\citep{RWeka:Urbanek:2010}, \pkg{RWeka} provides R ``interface
generators'' for setting up interface functions with the usual ``R look
and feel'', re-using Weka's standardized interface of learner classes
(including classifiers, clusterers, associators, filters, loaders,
savers, and stemmers) with associated methods.
\cite{RWeka:Hornik+Buchta+Zeileis:2009} discuss the design philosophy of
the interface, and illustrate how to use the package.

Here, we discuss several important items not covered in this reference:
Weka packages, persistence issues, and possibilities of using Weka's
visualization and GUI functionality.

\section{Weka packages}

On 2010-07-30, Weka 3.7.2 was released, with a new package management
system its key innovation.
% (\url{http://forums.pentaho.com/showthread.php?77634-New-Weka-3.4.17-3.6.3-and-3.7.2-releases}).
This moves a lot of algorithms and tools out of the main Weka
distribution and into ``packages'', featuring functionality very similar
to the R package management system.  Packages are provided as zip files
downloadable from a central repository.  By default, Weka stores
packages and ther information in the directory specified by the
environment variable \verb|WEKA_HOME|; if this is not set, the
\file{wekafiles} subdirectory of the user's home directory is used.
Inside this directory, subdirectory \file{packages} holds installed
packages (each contained its own subdirectory), and \file{repCache}
holds the cached copy of the meta data from the central package
repository.  Weka users can access the package management system via the
command line interface of the \verb|weka.core.WekaPackageManager| class,
or a GUI.  See e.g.\
\url{http://weka.wikispaces.com/How+do+I+use+the+package+manager\%3F} for
more information.

For the R/Weka interface, we have thus added \verb|WPM()| for
manipulating Weka packages from within R.  One starts by building (or
refreshing) the package metadata cache via
\begin{Schunk}
\begin{Sinput}
> WPM("refresh-cache")
\end{Sinput}
\end{Schunk}
and can then list already installed packages 
\begin{Schunk}
\begin{Sinput}
> WPM("list-packages", "installed")